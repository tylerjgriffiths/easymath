\documentclass[11pt]{memoir}

%% Titling information
\title{\Huge \textsc{Easymath}\\ \huge\emph{Slightly faster mathematical typesetting}}
\author{Tyler Griffiths}
\date{}

%% Packages
\usepackage{booktabs}
\usepackage{easymath}
\usepackage[labelfont=bf, margin=2cm]{caption}

%% Appearance
\nouppercaseheads

%% Increase spacing in tables
\def\arraystretch{1.4}
\renewcommand{\thesection}{\arabic{section}}
\renewcommand{\thetable}{\thesection.\arabic{table}}

\begin{document}

\maketitle

\tableofcontents

\section{Introduction}

This {\LaTeX} package provides a number of commands to make it easier to ``live-\TeX'' mathematically-heavy lectures. It generally reflects the sorts of mathematics I've had to type during the course of an undergraduate Chemistry degree, taking an additional module in scientific computing (with a focus on partial differential equations).

Every command is wrapped in \verb=\ensuremath= and so can be used in prose directly, without entering \verb=$math mode$=.

\section{Symbols}

This package defines a few symbols, largely for later internal use.

\begin{table}
\centering
\caption{Symbols provided by \textsc{Easymath}.}
\begin{tabular}{@{}lll@{}}
\toprule
Command & Example & Description \\
\midrule
\verb=\D= & \D & Differential operator \\
\verb=\del= & \del & Partial differential operator \\
\verb=\E= & \E & Euler's number \\
\verb=\Ey=, \verb=\egy=, \verb=\energy= & \Ey & Energy \\
\verb=\ham= & \ham & Hamiltonian operator \\
\bottomrule
\end{tabular}

\end{table}

\section{Brackets}

Several types of brackets are provided.

\begin{table}
\centering
\caption{Brackets provided by \textsc{Easymath}.}
\begin{tabular}{@{}lll@{}}
\toprule
Command & Example & Description \\
\midrule
\verb=\br= & \br{abc} & Parentheses\\
\verb=\abr= & \abr{abc} & Angle brackets \\
\verb=\cubr= & \cubr{abc} & Curly braces\\
\verb=\sqbr= & \sqbr{abc} & Square brackets \\
\verb=\bra= & \bra{\phi} & Dirac ``bra'' \\
\verb=\ket= & \ket{\psi} & Dirac ``ket'' \\
\verb=\braket= & \braket{\phi}{\psi} & Dirac ``braket'' \\
\verb=\bramket= & \bramket{\phi}{A}{\psi} & Dirac ``braket'' (mnemonic: \texttt{m}\emph{iddle})\\
\bottomrule 
\end{tabular}
\end{table}

\section{Functions}

Here are provided some common functions and useful snippets for writing them.

\begin{table}
\centering
\caption{Function commands provided by \textsc{Easymath}.}
\begin{tabular}{@{}lll@{}}
\toprule
Command & Example & Description\\
  \midrule
  \verb=\tx= & $\energy_\tx{total}$ & A shorter alias for \verb=\text=. Does not require \{\} in su\{b,per\}script.\\
\verb=\of= & $f\of{x}$ & Correctly-spaced, correctly-sized, function argument. \\
\verb=\upf= & $\upf{abc}{x}$ & \texttt{Up}right multiletter \texttt{f}unction with argument. \\
  \verb=\ex= & \ex{i\pi} & Exponential function with argument. \\
  \verb=\inv= & \inv{\rho} & Inverse. \\
  \verb=\is= & $y \is ax^2$ & Algebraic assignment.\\
\bottomrule
\end{tabular}
\end{table}

\section{Differentials}

Much of this package aims to make typesetting of differential equations faster. Utilities are provided for both partial and ordinary differentials.

\begin{table}
\centering
\caption{Differential functions provided by \textsc{Easymath}.}
\begin{tabular}{@{}lll@{}}
\toprule
Command & Example & Description \\
\midrule
\verb=\diffn= & \diffn{y}{x}{4} &$n$th order ordinary differential.\\
\verb=\diff= & \diff{y}{x} & First-order differential.\\
\verb=\ddtn= & \ddtn{y}{6} & $n$th order ordinary differential with respect to time. \\
\verb=\ddt= & \ddt{y} & First-order ordinary differential with respect to time. \\
\verb=\pdiffn= & \pdiffn{y}{x}{2} & $n$th order partial differential.\\
\verb=\pdiff= & \pdiff{y}{x} & First-order partial differential.\\
\verb=\pdtn=, \verb=\pddtn= & \pdtn{T}{2} & $n$th order partial differential with respect to time.\\
\verb=\pdt=, \verb=\pddt= & \pdt{\Gamma} & First-order partial differential with respect to time.\\
\bottomrule
\end{tabular}
\end{table}

\section{Integrals}

A few commands are provided to make it easier to type integrals.

\begin{table}
\centering
\caption{Integral functions provided by \textsc{Easymath}.}
\begin{tabular}{@{}lll@{}}
\toprule
Command& Example& Description\\
\midrule
\verb=\lint= & \lint{a}{b}{f\of x}{x} & Limited integral.\\
\verb=\uint= & \uint{g\of r}{r} & Unlimited integral.\\
\bottomrule
\end{tabular}
\end{table}

\end{document}


%%% Local Variables:
%%% mode: latex
%%% TeX-master: t
%%% End:
